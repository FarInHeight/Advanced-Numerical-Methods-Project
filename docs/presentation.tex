\documentclass[10pt]{beamer}
\definecolor{title}{RGB}{1, 45, 112}
\definecolor{body}{RGB}{218, 233, 255}
\usetheme[progressbar=frametitle]{metropolis}
\usecolortheme{wolverine}
\usefonttheme{professionalfonts}
\setbeamertemplate{navigation symbols}{}
\setbeamercolor{block title}{bg=title,fg=white}
\setbeamercolor{block body}{bg=body!40,fg=black}
\setbeamertemplate{section in toc}[square]
\setbeamertemplate{subsection in toc}[square]

\usepackage{libertinus}

\usepackage{bm}
\usepackage{amsmath, amssymb}
\usepackage{ragged2e}
\justifying

\usepackage[backend=biber, style=verbose]{biblatex}
\addbibresource{references.bib}
\nocite{*}

\title{Singular Value Decomposition}
\subtitle{An application to Big Data}
\author{Davide Sferrazza}
\institute{Università degli Studi di Palermo}
\date{\today}

\def\titlepage{%
  \usebeamertemplate{title page}%
}

\begin{document}
\maketitle

\begin{frame}
    \frametitle{Outline}
    \tableofcontents
\end{frame}

\section{Singular Value Decomposition}
\subsection{What is it?}

\begin{frame}
    \frametitle{Definition of SVD}
    \begin{theorem}
        Given a matrix $A \in \mathbb{R}^{m \times n}$, it can always be found a decomposition such that
        $$A = U \Sigma V^T$$
        where $U \in \mathbb{R}^{m \times m}$, $V \in \mathbb{R}^{n \times n}$ and $\Sigma \in \mathbb{R}^{m \times n}$.

        $U$ and $V$ are two orthogonal matrices and $\Sigma$ is a diagonal matrix, namely:
        $$ ( \Sigma )_{ij} = 
            \begin{cases}
                0, & i \ne j \\
                \sigma_i, & i = j
            \end{cases} 
        $$
        where $\sigma_1 \ge \sigma_2 \ge \ldots  \ge \sigma_p \ge 0$, $p = \min\{m, n\}$.
    \end{theorem}
\end{frame}

\begin{frame}
    \frametitle{Definition of SVD}
    The non-zero entries of $\Sigma$, denoted by $\sigma_i$, are called \textit{singular values}.

    They are arrenged in a nonincreasing order by convention.

    The column vectors $\bm{u_i}$ of $U$ are called \textit{left singular vectors} and those $\bm{v_i}$ of $V$ are called \textit{right singular vectors}. \bigskip
    
    Since in general $m \ne n$, we have: 
    $$ A = \sum_{i = 1}^{p} \bm{u_i} \sigma_i \bm{v_i}^T $$

\end{frame}

\begin{frame}{Definition of SVD}
    \begin{theorem}
        If for some $r$ such that $ 1 \le r < p $ we have
        $$ \sigma_1 \ge \ldots \ge \sigma_{r} > \sigma_{r + 1} = \ldots = \sigma_p = 0  $$
        then
        \begin{itemize}
            \item $rank(A) = r$
            \item $ A = \sum \limits_{i = 1}^{r} \bm{u_i} \sigma_i \bm{v_i}^T $
        \end{itemize}
    \end{theorem}

    This means that all other $p-r$ dimensions of matrix $A$ are linear combinations of the first $r$.
\end{frame}

\begin{frame}{Definition of SVD}
    \begin{block}{Lower rank approximation}
        Let $A \in \mathbb{R}^{m \times n}$ be a matrix whose rank is $rank(A) = r$. \\
        If for a fixed integer value $k < r$ we define
        \begin{equation}
            \label{eq:lower_rank}
            A_k = \sum \limits_{i = 1}^{k} \sigma_i \bm{u_i}  \bm{v_i}^T 
        \end{equation}
        and
        $$ \mathcal{B} = \left\{ B \in \mathbb{R}^{m \times n} : rank(B) = k \right\}$$
        then 
        $$ \min_{B \in \mathcal{B}} \left\lVert A - B \right\rVert _2 = \left\lVert A - A_k \right\rVert _2 = \sigma_{k + 1} $$
    \end{block}
        
    This result tell us that $A_k$ represents the best approximation (considering the \textit{spectral norm}) of rank $k$ of matrix $A$.
\end{frame}

\subsection{How can singular values be computed?}

\begin{frame}{Singular values computation}
    To compute the singular values, consider the transponse of $A$ given its decomposition:
    $$ A^T = (U \Sigma V^T)^T = V \Sigma^T U^T$$
    The symmetric matrix $A^T A$ is equal to:
    $$ A^T A = ( V \Sigma^T U^T )(U \Sigma V^T) = V \Sigma^T \Sigma V^T$$
    Furthermore, this equation can be written as:
    $$ A^T A V = V \Sigma^T \Sigma $$
    This means that the diagonal entries of the square matrix $ \Sigma^T \Sigma $, which are the square of the singular values, are the eigenvalues of matrix $A^T A$ and $ V $ is the matrix of eigenvectors.
\end{frame}

\begin{frame}{Singular values computation}
    Similarly, consider the product of $A A^T$. It is equal to:
    $$ A A^T = (U \Sigma V^T)( V \Sigma^T U^T ) = U \Sigma \Sigma^T U^T$$
    Which means that:
    $$ AA^T U = U \Sigma \Sigma^T $$
    Hence $U$ is the matrix of eigenvectors of $AA^T$. \bigskip

    Since $rank(A) = r$, only the first $r$ eigenvalues of $AA^T$ and $A^T A$ are non-zero.
\end{frame}

\section{References}
\begin{frame}[allowframebreaks, noframenumbering]{References}
    \printbibliography
\end{frame}

\end{document}