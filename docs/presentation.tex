\documentclass{beamer}
\usepackage{bm}
\usetheme{Rochester}
\usecolortheme{wolverine}
\usefonttheme{professionalfonts}

\title{Singular Value Decomposition}
\subtitle{An application to Big Data}
\author{Davide Sferrazza}
\institute{Università degli Studi di Palermo}
\date{\today}

\begin{document}

\begin{frame}
    \titlepage
\end{frame}

\begin{frame}
    \frametitle{Outline}
    \tableofcontents
\end{frame}

\section{Singular Value Decomposition}
\subsection{What is it?}

\begin{frame}
    \frametitle{Definition of SVD}
    \begin{theorem}
        Given a matrix $A \in \mathbb{R}^{m \times n}$, it can always be found a decomposition such that
        $$A = U \Sigma V^T$$
        where $U \in \mathbb{R}^{m \times m}$, $V \in \mathbb{R}^{n \times n}$ and $\Sigma \in \mathbb{R}^{m \times n}$.

        $U$ and $V$ are two orthogonal matrices and $\Sigma$ is a diagonal matrix, namely:
        $$ ( \Sigma )_{ij} = 
            \begin{cases}
                0, & i \ne j \\
                \sigma_i, & i = j
            \end{cases} 
        $$
        where $\sigma_1 \ge \sigma_2 \ge \ldots  \ge \sigma_p \ge 0$, $p = \textrm min\{m, \, n\}$.
    \end{theorem}
\end{frame}

\begin{frame}
    \frametitle{Definition of SVD}
    The non-zero entries of $\Sigma$, denoted by $\sigma_i$, are called \textit{singular values}.

    They are arrenged in a nonincreasing order by convention.

    The column vectors $\bm{u_i}$ of $U$ are called \textit{left singular vectors} and those $\bm{v_i}$ of $V$ are called \textit{right singular vectors}. \bigskip
    
    Since in general $m \ne n$, we have: 
    $$ A = \sum_{i = 1}^{p} \bm{u_i} \sigma_i \bm{v_i}^T $$

\end{frame}

\begin{frame}
    \frametitle{Definition of SVD}
    
\end{frame}

\subsection{How can singular values be computed?}

\begin{frame}
    \frametitle{Singular values computation}
    To compute the singular values, consider the transponse of $A$ given its decomposition:
    $$ A^T = (U \Sigma V^T)^T = V \Sigma^T U^T$$

    
\end{frame}

\end{document}